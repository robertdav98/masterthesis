\chapter{Evaluation}
\label{cha:evaluation}

\section{Festlegen der Evaluations Metriken}
Folgende Metriken werden im folgenden Betrachtet:
\begin{itemize}
	\item Laufzeit
	\item Sicherheit
\end{itemize}
\subsection{Metrik - Laufzeit}
Um die Laufzeit der Operationen zu messen wurde ein Python-Script entwickelt, welches die Operationen 'createIdentity', 'getIdentity', 'createClaim', 'getClaimQR' und 'revokeClaim' jeweils 100 mal ausführt. Jeder Aufruf startet einen eigenen Subprozess mit dem Python Framework 'subprocess', welcher mit dem Framework 'timeit' gemessen wird. Gemessen wird der Durchschnitt und der Median bei jeweils 100 Ausführungen einer Operation. Die gemessenen Werte sehen wie folgt aus:

\begin{center}
	\begin{tabular}{ c c c }
		Operation\_Name 	& Durchschnitt	& Median 		\\
		createClaim 		& 1.2098522 	& 1.0454559 	\\
		createIdentity 		& 1.1128388 	& 1.1019622  	\\
		getClaimQR 			& 1.5566253 	& 1.2646243 	\\
		getIdentity			& 1.1031939 	& 1.0355741 	\\
		revokeClaim 		& 1.4271623 	& 1.526894 		\\
	\end{tabular}
\end{center}
Es ist zu erkennen, dass der Median in jedem Fall geringer ausfällt als der Durchschnitt. Dies muss damit zusammenhängen, dass ein Teil der Aufrufe unverhältnismäßig länger dauern.\\

Der folgenede Code zeigt wie die Messung der Laufzeit stattgefunden haben.

\begin{lstlisting}[language=python]
	def runMeasurement(command, cache):
		runs = 100
		key = command.split()[3]  # e.g. "node app -a createIdentity" -> createIdentity
		try:
			runtime = timeit(stmt="subprocess.call('{}')".format(command), setup="import subprocess", number=runs, )
			durations = repeat(stmt="subprocess.call('{}')".format(command), setup="import subprocess", number=1, repeat=runs)
		
			cache[key + "_median"] = calcMedian(durations)
			runtime = checkisFloat(runtime)
			if runtime:
				cache[key + "_average"] = runtime
		except:
			pass  # the error here can be ignored
			
	cache = {}
	runMeasurement(create_identity_command, cache)
	runMeasurement(get_identities_command, cache)
	runMeasurement(create_claim_command.format("Truck", "2000", first_identifier), cache)
	runMeasurement(get_claim_command.format(first_claim, first_identifier), cache)	
\end{lstlisting}	

\section{Sicherheitsevaluierung}
Die Komponente, die den größten Schutz in der Implementierung benötigt ist der Issuer. Das SSI-Konzept ist abhängig davon, wie vertrauenswürdig Letzterer ist. Daher findet eine Security-Analyse nach dem STRIDE-Modell \cite{ID59} statt..

\subsection{Spoofing}
\label{Spoofing}
Dieser Angriff beschreibt unberechtigten Zugriff auf die Komponente. Hiergegen wurde implementiert, dass der Zugriff (der über ngrok stattfindet) nur authentifiziert passieren kann: --basic-auth 'ngrok:issecure' --basic-auth='username:password'. Zusätzlich besitzt die verwendete REST-API eine Basic-Authentifikation. Daher ist diese Komponente doppelt geschützt. Es wird stark empfohlen, dass die Passwörter den Richtlinien des BSI entsprechen \cite{ID60}.

\subsection{Tampering}
Dieser Angriff beschreibt die ungewollte Manipulation von Daten. Prinzipiell gibt können zwei Komponenten Daten verändern:
\begin{enumerate}
	\item Der Holder: Als Besitzer der Daten hat der Holder die Möglichkeit seine Credentials zu löschen. Dieser Prozess würde über die Applikation passieren, welche zunächst über ein potentielles Password des Geräts und zum anderen durch das Passwort der PolygonId-App geschützt ist.
	\item Der Issuer: Der Issuer kann ebenso Credentials revoken (widderufen). Aber wie bereits im Kapitel \ref{Spoofing} beschrieben ist der Issuer geschützt.
	\item Der Verifier: Hat keine Möglichkeit Credentials zu modifizieren und ist daher nicht weiter zu betrachten
\end{enumerate}

\subsection{Repudiation}
\label{Repudiation}
Dieser Angriff beschreibt, dass ein Nutzer eine Aktivität abstreiten kann. Gegen diese Attacke schützt die Blockchain, die jede Transaktion abspeichert. Durch das Monitoring-Tool von Alchemy oder im Blockchain-Browser können diese Transaktionen betrachtet werden. Eine solche Transaktion kann wie folgt aussehen:

\begin{lstlisting}[language=json,firstnumber=1]
{
	"jsonrpc": "2.0",
	"id": 0,
	"method": "eth_call",
	"params": [
	{
		"from": "0x0000000000000000000000000000000000000000",
		"to": "0x134b1be34911e39a8397ec6289782989729807a4",
		"data": "0x7c1a66de0a79f724bb72300544255781fc350952acb21cb77ea9a719c8eebb7d1a055ad0"
	},
	"latest"
	]
}
\end{lstlisting}	
Es ist zu sehen, dass sowohl übertragene Daten, als auch involvierte Adressen gespeichert werden. Das Schreiben ist ebenso kryptographisch geschützt.

\subsection{Information disclosure}
\label{disclosure}
Dieser Angriff beschreibt das ungewollte Veröffentlichen von Daten. In diesem Prototypen gibt es vier Typen von Daten:
\begin{enumerate}
	\item Daten im Issuer: Diese Daten sind streng geheim und werden unter anderem in '.env' Dateien oder in einem Vault gespeichert. Darunter sind private Schlüssel, API-Schlüssel für Alchemy, Nutzernamen und Passwörter für UI und API. Zusätzlich liegen alle diese Daten in einer virtuellen Maschine, die mit 2-Faktor-Authentifizierung und einem 20-Stellen Root-Passwort geschützt sind.
	\item Daten in der Blockchain: Diese Daten sind bereits öffentlich.
	\item Daten im Holder: Die Daten werden in der Wallet-App gespeichert. Darunter sind private Schlüssel und die Credentials, welche durch ein Passwort in der Polygon-App geschützt sind.
	\item Daten im Verifier: Der Verifier benötigt lediglich Zugang zu den Schemas, welche öffentlich gespeichert sind, und einen RPC-Node, welcher in Umgebungsvariablen lokal gespeichert ist.
\end{enumerate}
Daher lässt sich erkennen, dass private Daten sicher gespeichert sind und nicht veröffentlicht werde können.

\subsection{Denial of service}
Denial of Service beschreibt eine Attacke, in die komplette Software oder Teile davon ungewollt außer Betrieb genommen werden. Dieser Angriff findet in dem Prototypen nur Anwendung in dem Verifier, da andere Komponenten entweder lokal sind, oder nur der Nutzer Zugriff hat. Im Verifier könnte ein Angreifer probieren den RPC-Node zu überlasten. Jedoch hat Alchemy hiergegen Mechanismen entwickelt \cite{ID61}.

\subsection{Evelation of priviledge}
Dieses Konzept beschreibt, dass ein Nutzer ungewollt seine Rechte auf ein höheres Level setzten kann, um autorisiert zu sein neue Aktivitäten auszuführen. Diese Art der Attacke findet jedoch keine Anwendung, da es keine Levels an Rechten gibt.

\subsection{Zusammenfassung - STRIDE}
\begin{center}
	\begin{tabular}{ c c }
		KATEGORIE & Abgesichert? \\
		Spoofing & \checkmark \\
		Tampering (Manipulation) & \checkmark \\
		Repudiation (Nichtanerkennung) & \checkmark\\
		Information disclosure (Veröffentlichung von Informationen) &\checkmark\\
		Denial of service & \checkmark\\
		Evelation of priviledge (Erhöhung von Rechten) & \checkmark \\
	\end{tabular}
\end{center}
Es ist zu erkennen, dass der Prototyp sicher ist im Rahmen des STRIDE-Modells.

\section{Beantworten der Forschungsfragen}
Im Folgenden werden die in Kapitel \ref{forschungsfragen} gestellten Fragen beantwortet:



\begin{enumerate}
	\item Wie erfolgt die Speicherung verschiedener Typen an Daten innerhalb eines Identitätsmanagementsystems?
	\begin{enumerate}
		\item Wie kann die technische Machbarkeit erreicht werden, Daten sowohl öffentlich zugänglich als auch privat zu speichern? - Es ist möglich Daten privat und öffentlich zu speichern. Dies passiert zum einen bei Shocard in dem Prozess des Bootstrapping (siehe \ref{shocard}) und zum andere bei zk-Proofs in PolygonId. Jedoch ist es zu empfehlen, private Daten wie Credentials auch privat zu speichern, wie es bereits in verschiedenen Lösungen angewandt wird.
		
		\item Welche kryptografischen Technologien, wie zum Beispiel Hashing oder Verschlüsselung, sind optimal für die sichere Speicherung von Identitätsdaten geeignet? - Sowohl Hashing als auch (asymmetrische) Verschlüsslung werden für die Speicherung von Identitätsdaten verwendet, je nachdem welche Lösung angewandt wird.
		
		 
		\item Welche Sicherheitsmaßnahmen sind im Falle einer Datenkompromittierung zu ergreifen, und welche Wiederherstellungsoptionen sind verfügbar? - Sollte der Nutzer den privaten Schlüssel seines Identity-Wallets verlieren, so existiert eine Recovery-Option, bei welcher der User - die bei der Erstellung angezeigten - Passwörter erneut angeben muss. Sollte der Nutzer den privaten Schlüssel an einen Angreifer preisgeben, so ist der Wallet unbrauchbar.
		
		\item Wie lässt sich die sichere und effektive Ungültigmachung von Informationen (Revokation) gewährleisten? - Die Revokation kann auf drei Weisen geschehen.
		
		\begin{itemize}
			
			\item Es kann bei der Credential-Zuweisung eine maximale Gültigkeitsdauer angegeben werden. Läuft diese aus, so wird der Credential widerrufen
			
			\item als Issuer besteht die Möglichkeit Credentials zu wiederrufen
			
			\item als Holder besteht die Möglichkeit Credentials aus dem Identity-Wallet zu löschen und somit zu widerrufen
			
		\end{itemize}
		 
	\end{enumerate}
	
	\item Welchen Nutzer oder Service-Mehrwert generiert ein dezentrales Identitätsmanagementsystem basierend auf DLT?
	\begin{enumerate}
		
		\item Welche konkreten Vorteile ergeben sich für Nutzer und Online-Dienste durch die Implementierung des vorgeschlagenen Identitätsmanagementsystems? - Als Vorteil entpuppt sich für den Nutzer, dass er Herrscher seiner Daten ist und diese eigenständig verwalten kann. Dazu gehört, dass der Nutzer selber entscheiden kann, welche Daten er freigibt. Als Vorteil ergibt sich für einen Online-Dienst, dass er keine eigene personenbezogenen Daten speichern muss und Attribute von Nutzern überprüfen kann, die beispielsweise zum Kauf berechtigen oder Vergünstigungen anbieten.
		
		\item Inwiefern trägt das System zur Lösung der Problematik von Fake-Usern bei? - Ja das Problem der Fake-User kann hiermit gelöst werden. Beispielsweise kann bei der Registrierung die Verifizierung eines Credentials erfolgen, wobei dieser Credential nur von einer autorisierten Instanz ausgestellt werden kann.
	\end{enumerate}
	
	\item Wie erfolgt die Identitätszuordnung innerhalb des Identitätsmanagementsystems?
	
	\begin{enumerate}
		\item Wie kann eine verlässliche Zuordnung von Identitäten zu Personen auf hohem Sicherheitsniveau erreicht werden? - Diese Aufgabe liegt bei dem Issuer. Dieser muss sicherstellen, dass er keine Credentials zuweist an Autoritäten, denen er nicht vertrauen kann. Um Vertrauen aufzubauen kann beispielsweise  ein Know-Your-Customer-Prozess durchgeführt werden.
		
		\item Welche Ansätze können entwickelt werden, um das Problem zu lösen, dass Nutzer möglicherweise verschiedene Identitäten auf verschiedenen Plattformen verwenden möchten? - Dieser Faktor stellt sich nicht als Problem, sondern sogar als Feature heraus. Nutzer (auch Issuer) können mehrere DIDs besitzen. Bei Sovrin (siehe \ref{sovrin}) wird pro Interaktion eine neue DID verwendet, um maximale Privatsphäre zu gewährleisten.
		
	\end{enumerate}
	
	\item Welche technologischen Spezifika sind im Bezug auf die Blockchain als DLT im Identitätsmanagementsystem zu entscheiden?
	\begin{enumerate}
		\item Welcher Konsensus-Algorithmus ist am besten geeignet, um die Anforderungen des entwickelten Identitätsmanagementsystems zu erfüllen? - Proof-of-Stake stellte sich als passender Algorithmus herraus (siehe \ref{PolygonPOS})		
		\item Wie können private Schlüssel sicher gespeichert werden, um unbefugten Zugriff zu verhindern? - Private Schlüssel des Issuers befinden sich in einem Vault (siehe \ref{vault}) und private Schlüssel des Users befinden sich in der Applikation des Mobilgeräts.
		\item Inwiefern können Identitätsdokumente erfolgreich als NFTs gespeichert werden, und welche Implikationen ergeben sich daraus? - Ja, die Credentialausstellung ist leicht erweiterbar mit der Zustellung eines NFTs. Dies könnte sowohl on-chain im smartcontract geschehen oder in der REST-API des Issuers. Beide Komponenten sich beliebig erweiterbar.
		
		\item Welche Rolle spielen Zero-Knowledge-Proofs bei der Entwicklung eines sicheren Identitätsmanagementsystems? - ZK-Proofs werden vom Holder erstellt und an den Verifier gesendet, welcher diese überprüfen kann. Dies hat den Vorteil, dass der Verifier einen Beweis für den Besitzt eines Claims erhält ohne Details zu kennen.
	\end{enumerate}
\end{enumerate}

\section{Erfüllung der Anforderungen}
Die in Kapitel \ref{cha:anforderungsanalyse} gestellten Anforderungen werden die folgt erfüllt:
\begin{enumerate}
	\item Widerruf: Wird durch den Holder und Issuer ermöglicht
	\item Überprüfbarkeit: Der Holder kann zk-Proofs ausstellen, die der Verifier überprüfen kann
	
	\item Selektive Veröffentlichung: Wird zum einen durch 'selective disclosure' und zum anderen durch speichern von Credentials im öffentlichen Profil ermöglicht.
	\item Vertraulich: Mittels Passwörter und kryptographischen Verfahren ist diese Anwendung vertraulich
	\item Non-Replay: Jede Transaktion/Operation erfolgt authentifiziert und ist daher nicht reproduzierbar
	\item Nichtabstreitbarkeit: Ist gegeben (siehe \ref{Repudiation})
	\item Diebstahlschutz: Ist gegeben (siehe \ref{Spoofing} und \ref{disclosure}
	\item W3C-Standard : Wird erfüllt \footnote{auch ersichtlich im Schema 'xsd: http://www.w3.org/2001/XMLSchema'}
\end{enumerate}
Es wird also ersichtlich, dass alle Anforderungen erfüllt wurden.

