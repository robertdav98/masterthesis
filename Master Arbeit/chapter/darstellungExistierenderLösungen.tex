\chapter{Darstellung existierender Lösungen}
\label{cha:existierende Lösungen}

\section{Luniverse}
Luniverse \cite{ID27} ist eine Firma, die im Mai 2018 gegründet wurde und BaaS (Blockchain as a Service) anbietet, indem ein umfassendes Portfolio an Blockchain Lösungen angeboten wird, die verschiedene Problematiken der "Blockchain-Umstellung" lösen. 2022 bedient Luniverse nach eigenen Angaben über 2000 Firmenkunden. Dabei gibt es vier Kategorien, die allesamt das Sidechain-Konzept implementieren. \\
Dabei ist eine Sidechain \cite{ID29} \cite{ID33} eine separate Blockchain, die 
\begin{itemize}
	\item parallel zur ursprünglichen Blockchain läuft. Dies bedeutet, dass die Knoten der Hauptkette und Knoten der SideChain unabhängig voneinander agieren.
	\item bidirektional mit dem Mainnet verknüpft ist. Dies hat den Vorteil, dass Informationen zwischen der SideChain und der Hauptkette in beide Richtungen transportiert werden können. Letzteres passiert beispielsweise dadurch, dass die Hauptkette Vermögenswerte blockiert ('lockt'), um diese auf der SideChain zugänglich zu machen. Diese Aktion kann in beide Richtungen erfolgen.
	
	\item ihren eigenen Konsensus-Algorithmus hat. Ein Beispiel hierfür ist, dass die Hauptkette Proof-of-Stake verwendet und die SideChain Proof-of-Authority.
	\item ihre Sicherheit \textbf{"nicht"} von der Elternkette erbt (hängt unter anderem mit dem oberen Punkt zusammen)
	\item erweiterte Funktionen implementieren kann. So ist es möglich, dass die SideChain beispielsweise 'zero-knowledge-proofs' anbietet auch wenn die Hauptkette diese Funktion nicht besitzt.
	\item neue Anwendungsfälle umsetzt. Während die Hauptkette evtl. nur ein Gerüst mit den Standard-Funktionen anbietet (Transaktionen von Vermögenswerten) kann es sein, dass die SideChain sich auf Supply-Chain-Management speziallisiert oder Finanzoperationen abbildet (Vermögenswerte leihen, Zinsen, etc)
\end{itemize}
ohne dabei die Elternkette zu beeinträchtigen.

\begin{enumerate}
	\item Luniverse NFT \\
	Mit diesem Dienst wird versprochen, dass die 'Luniverse-Multichain-NFTs' sowohl die Emissionskosten entfernen als auch die Umweltprobleme lösen. Dabei finden die Transaktionen zunächst nur auf der Luniverse-Sidechain statt, wobei die NFTs auch auf das Ethereum-Ökosystem übertragen. Es wird das Erstellen ('minten') von NFTs nach dem ERC721 Interface, ein Marktplatz für NFTs, geteiltes Eigentum von NFT, eine Datenbank für Metadaten und vieles mehr angeboten. Hierbei wird eine sehr hohe Energieeffizienz durch das Verwenden des LPOA-Konsensus-Algorithmus garantiert, welcher keine Transaktionskosten generiert.
	Weitere Vorteile der Luniverse-Sidechain sind: Mehr Transaktionen pro Sekunde, Anbieten einer REST-API, ein CLI und viele weitere
	
	\item Loyalty Point \\
	Dieser Dienst ist das Blockchain-Äquivalent von Treuepunkten. Es wird angeboten Treuepunkte dezentral zu organisieren, wodurch eine bessere Kundenakquise, eine stärkere Bindung von Kunden und das Übertragen von Treuepunkten zwischen Unternehmen angeboten wird.
	
	\item Trace \\
	Dieser Dienst ist dafür zuständig jegliche Aktivität auf der Blockchain aufzuzeichnen, um im Anschluss Datenintegrität, Verlaufsaufzeichnungen in Zeitreihendatenbanken und Datenverfolgung anzubieten.
	
	
	\item DID \\
	Unter diesem Dienst wird der ganze Prozess rund um DID's, DID-Dokumenten, Claims, etc. abgebildet. Es wird eine REST-API zur Verfügung gestellt, jedoch wird auch das UI benötigt. Beispielsweise werden Templates für Credentials oder für Verifier im UI erstellt.
	Davon abgesehen existieren HTTP-Endpunkte für das Überprüfen der Stati von widerrufenen Credentials, das Ausstellen, das Widerrufen und das Verifizieren von Credentials. Endpunkte für das Generieren von DIDs, DID-Docs und Ähnlichem stehen nicht zur Verfügung.
	
\end{enumerate}

Bemerkenswert ist, dass die oben genannten Dienste sich gegenseitig ergänzen. So kann die SSI oder NFT Aktivität mittels dem Trace-Dienst analysiert werden. Vor allem NFT's, die auch das Konzept des Eigentums implementieren könnten für ein umfangreiches IDP von Bedeutung sein. \\
Auf technischer Sicht arbeitet die API mit nur einem Parameter in den Anfragen: \textsl{didProjectId}. Dieser ist ein Pfad zu einer in der UI erstellen Ressource (Beispielsweise Templates für das Anfragen oder Verifizieren von Credentials). Davon abgesehen verwendet Luniverse eigene DID-Methoden, was an folgender beispielhaften Holder-DID deutlich wird: 'did:ethr:lunvs:0x1111111111111111111111111111111111111111' 

\section{Dock}
'Dock Certs (certs.dock.io)' ist ein Webdienst, der es ermöglicht Credentials anzufragen, Templates für Credentials, Templates für die Verification von Credentials, das Erstellen von Issuer-Profilen und das visuelle Gestalten von VC im PDF-Format.\\
Um Credentials anzufragen, zu erstellen oder zu verifizieren muss zunächst ein Template erstellt werden. Dies passiert durch das UI. Hier kann man entweder Templates als JSON importieren (auch mittels öffentlichen URL's von \textsl{public ledgern} wie IPFS [https://ipfs.tech]) oder im UI jedes Attribut einzeln konfigurieren. Anschließend muss ein Issuer-Profil erstellt werden. Dieses besteht aus einem öffentlichen Namen, einer optionalen Beschreibung und einem 'DID-Type', welcher entweder vom Typ 'dock' ist oder 'polygonid' (siehe nächste Lösung).
Im Anschluss können Credentials ausgestellt werden, die zuvor erstellen Templates entsprechen müssen. Daraufhin kann der Nutzer alle Empfänger des VC entweder manuell eintragen oder als Excel importieren lassen. Hierbei kann auch eine Email angeben werden, sodass der Rezipient eine Email erhält, um das VC zu akzeptieren. In der Email befindet sich ein QR-Code, welcher beispielsweise mit der PolygonID-App gescannt werden kann, wodurch das VC auf den Wallet übertragen wird.
Ebenso lassen sich im UI Verifikationstemplates erstellen, die beispielsweise festlegen welcher Claim (zum Beispiel eine Note) vorhanden sein muss.\\
Alle diese Funktionen lassen sich auch über die REST-API ausführen. Zusätzlich (und dies ist nicht über das UI möglich) lassen sich DID's und DID-Docks erstellen und Löschen, VC widerrufen, etc. \\
Es wird angegeben, dass alle Formate W3C-Standard-konform sind.

Anders als bei Luniverse handelt es sich bei Dock nicht um eine Sidechain, sondern eine eigenständige Blockchain. Dock wirbt damit eine Blockchain entwickelt zu haben, die für die dezentrale Identität optimiert ist \cite{ID30}. Der verwendete Konsensus-Algorithmus lautet GRANDPA (GHOST-based Recursive Ancestor Deriving Prefix Agreement). Eine wichtige Eigenschaft für diesen Kontext ist, dass die Blockproduktion und das Bestätigen von Transaktionen möglichst effizient gestaltet ist. Zudem werden Validato ren (oder auch Nominatoren/Staker) benötigt, die eine bestimmte Menge der Kryptowährung der Blockchain als Kaution hinterlegen müssen, um am Konsensprozess teilzunehmen. Ein Validator ist ein sog. "full node", also ein Knoten des Netzwerks der eine vollständige Kopie der gesamten Blockchain enthält. Hierbei arbeiten bis zu 50 Validatoren parallel, um die Integrität des Netzwerks zu bewahren. Zudem unterstützt die Blockchain das Konzept des Stakings, was bedeutet, dass Teilnehmer dem Netzwerk eine Menge an Kryptowährung hinterlegen, um das Netzwerk sicherer zu gestalten. Als Belohnung erhält der Teilnehmer an bestimmte Menge an 'Dock' (so lautet die Einheit der Kryptowährung).


\section{PolygonId}
PolygonId (https://polygon.technology/polygon-id) ist eine Plattform, die einem Entwickler die Möglichkeit gibt 'eine vertrauenswürdige und sichere Beziehung zwischen Nutzern und dApps (dezentrale Applikationen) zu bauen, die den Prinzipien von SSI und \textbf{privacy by default} folgen' \cite{ID31}.
'Privacy by default' beschreibt hierbei, dass Claims mittels ZK-proofs (zero-knowledge) überprüft werden können. 

\subsection{Polygon}
PolygonId dient als Identitätsinfrastruktur auf der Polygon-Blockchain, welche wiederum eine Layer-2-Lösung ist \cite{ID34}, was bedeutet, dass 
\begin{itemize}
	\item Ethereum (Layer 1) das übergeordnete Netzwerk ist
	\item Polygon direkt mit der Hauptkette (Ethereum Mainnet) verbunden ist und dessen Sicherheit und Konsensmechanismen nutzt
	\item Polygon darauf abzielt die Transaktionsfrequenz zu erhöhen, indem Transaktionen auf außerhalb der Hauptkette ausgelagert werden
	\item Polygon als Skalierungslösung zk-Rollups verwendet. Demnach wird eine große Menge an Transaktionsdaten auf der - vom Polygon-Framework zur Verfügung gestellten - Sidechain in ein Batch verpackt und mittels zk-Proofs auf der Hauptkette verifiziert, ohne die Details der Transaktionen zu kennen. Eine alternative zu zk-Rollups sind sog. 'optimistic rollups', was bedeutet, dass das Mainnet den Batch 'beglaubigt' \cite{ID35}. In dem Falle, dass sich Fehler oder Unstimmigkeiten innerhalb des Batches befinden kann eine sog. challange eingereicht werden, wodurch Transaktionen überprüft werden durch Teilnehmer des Netzwerks.
\end{itemize}
Zusätzlich bietet das Polygon weitere Mechanismen an, wobei im Folgenden nur solche genannt werden, die für das Implementieren von SSI-IDPs von Bedeutung sind oder werden könnten:
\begin{itemize}
	\item Polygon POS Chain \label{PolygonPOS} \cite{ID36}: Ist die Hauptkomponente der Polygon-Plattform und ist eine Proof-of-Stake Blockchain (zuvor auch Matic Network genannt). In der Kryptowährung Matic (MATIC) werden die Transaktionsgebühren gezahlt. Die über PolygonID getätigten Transaktionen finden hier statt und profitieren von Skalierbarkeit, hoher Geschwindigkeit, Benutzerfreundlichkeit und Konnektivität zu Ethereum. Auch wenn es sich hierbei um eine Sidechain handelt ist es bisher \textbf{nicht} möglich DIDs von Polygon auf Ethereum zu übertragen, was bedeutet, dass VP's nur innerhalb von Polygon existieren und andere Ethereum-basierende Lösung diese nicht verifizieren können. Am 21.Juli 2023 wurde angekündigt, dass durch eine Partnerschaft PolygonId auch auf der Ethereum-Blockchain verwendet werden kann \cite{ID36}.
	\item Polygon Bridge: Ist ein Mechanismus um Vermögenswerte (also Kryptowährungen) zwischen Ethereum und Polygon-Sidechains zu bewegen. Dies kann von Bedeutung sein, wenn beispielsweise beim Anfordern eines VC eine Gebühr bezahlt werden muss (was in der analogen Welt regelmäßig stattfindet) und die Vermögenswerte nicht auf einer Blockchain abgesperrt werden sollen.
	\item Zusätzlich gibt es mehrere Tools für Software-Entwickler eigene Sidechains zu implementieren (Polygon SDK) oder ein Testnetz, um Smartcontracts und ähnliches zu deployen und zu testen bevor auf dem Mainnet gearbeitet wird.
	
\end{itemize}
All die oben genannten Punkte sind für PolygonId von Bedeutung, da alle Schreib und Verifizierungsvorgänge (das Registrieren oder Updaten von DID's, Verifizieren von Claims, etc.) auf der Polygon-Blockchain stattfinden. \\

\subsection{Usecases und technische Daten über PolygonId}
\label{technischeDatenPolygon}
Nach eigenen Angaben sind Usecases von PolygonId
\begin{itemize}
	\item Digitale Demokratie (eine Person, eine Stimme)
	\item Passwordless Login
	\item private Zugangskontrolle
	\item weitergehende technische Features, wie 'Sybil-Proof"\footnote{Attacke wo Identitäten gefälscht werden - gefährlich bei Mehrheitsabstimmungen oder zum Verlangsamen des Netzwerks}-Protokolle oder Frameworks, die die Entwicklung von SSI-Applikationen vereinfachen.
\end{itemize}
\newpage
Auf technischer Sicht bietet PolygonId ein Framework um folgendes Dreieck zu implementieren:
\begin{figure}[H]
	\centering
	\includegraphics[scale=0.3]{media/polygonIddreieck_inv.png}
	\caption{Zusammenspiel von Issuer, Holder und Verifier auf technischer Sicht in PolygonId \cite{ID31}}
	\label{fig:meine-grafik}
\end{figure}

Es ist zu erkennen, dass Polygon SDK's für Entwickler zur Verfügung stellt um jeweils (Identity-) Holder, Verifier und Issuer zu implementieren. Für Wallets gibt es zudem eine Wallet-App. Für Issuer müssen Issuer Node gehostet werden, wodurch VC's ausgestellt oder entfernt werden können oder \textsl{Identity States} on-chain veröffentlicht werden können. Nach der Dokumentation \cite{ID38} muss der Node zusammen mit 

\begin{itemize}
	\item der Applikation
	\item einem 'Vault' für Key-Management-Services (Beispielsweise zum Speichern des privaten Schlüssels des Issuers)
	\item einem Cache-Service (Bespielsweise zum Zwischenspeichern von Templates, die einmalig von IPFS gedownloaded werden)
	\item einer Datenbank zum persistieren aller operativen Daten
\end{itemize}
\label{merkle}
Die Identitätsdaten bestehen aus folgenden drei Informationen, die jeweils in sog. 'Sparse Merkle Trees' gespeichert werden:
\begin{enumerate}
	\item \textbf{Claims Tree}: Ist ein Baum, der Informationen über ausgestellte Claims einer Identität \textbf{öffentlich} speichert.
	\item \textbf{Revocations Tree}: Ist ein Baum, der die Noncen (zuvor zufällig generierte Zahlen) einer Widerrufung von Claims \textbf{privat} speichert.
	\item \textbf{Roots Tree}: Speichert \textbf{öffentlich} die Historie der Wurzeln der Claim Trees.
\end{enumerate}
Der Identitätsstatus lässt sich somit wie folgt definieren:
\begin{align}
	IdState = Hash(ClR || ReR || RoR)
\end{align}
wobei:
\begin{itemize}
	\item Hash einer Hash-Funktion entspricht
	\item ClR der Wurzel des Claims Tree entspricht
	\item ReR der Wurzel des Revocation Tree entspricht
	\item ReR der Wurzel des Roots Tree entspricht
\end{itemize}
Der oben genannte Identitätsstatus wird als einziges Datum in der Blockchain gespeichert.

\subsection{zk-Proof von PolygonId}
Ein Feature von PolygonId, dass es von anderen SSI-Frameworks hervorhebt ist, dass es Verifikation mittels sog. \textsl{zero-knowledge-proofs} zur Verfügung stellt. Die Idee hierbei ist, dass dem Verifier die Richtigkeit eines Claims verifizieren kann \textbf{ohne} Informationen über den eigentlichen Claim zu erhalten. Als Beispiel wird folgender \textbf{nicht anonymisierter / nicht verschlüsselter/ selektiv freigegebener} Claim betrachtet:
\begin{lstlisting}[language=json,firstnumber=1]	
[...]
	"credentialSubject": {	
		"id": "did:example:456",	
		"degree": {		
			"type": "Gehalt",		
			"name": "ArbeitgeberName",		
			"frequence": "monthly",
			"value": 4000	
		}	
	}
[...]
\end{lstlisting}
Sollte nun beispielsweise ein Kreditinstitut (der Verifier) überprüfen wollen, ob der Kunde mehr als 2000€ verdient, so ist es in erster Linie irrelevant, ob die Person 5000€ oder 10000€ verdient und würde nur unnötig die Privatsphäre beeinträchtigen. Polygon bietet hierzu eine Lösung und implementiert ein zero-knowledge-Konzept, wo der Verifier über eine Query-Sprache lediglich über den Besitz des Credentials informiert wird ohne tatsächlich Informationen zu erhalten.
Die Anfrage kann on-chain und off-chain formuliert und sieht wie folgt aus (erstellt mit dem ZK-QueryBuilder\footnote{https://schema-builder.polygonid.me/query-builder} von PolygonId):

\begin{lstlisting}[language=json,firstnumber=1]	
[...]
    const proofRequest: protocol.ZKPRequest = {
		id: 1,
		circuitId: 'credentialAtomicQuerySigV2',
		query: {
			allowedIssuers: ['*'],
			type: 'EmployeeData',
			context: 'https://raw.githubusercontent.com/0xPolygonID/tutorial-examples/main/credential-schema/schemas-examples/employee-data/employee-data.jsonld',
			credentialSubject: {
				monthlySalary: {
					$gt: 2000,
				},
			},
		},
	};
[...]
\end{lstlisting}
Das Ergebnis der Anfrage ist ein Objekt, welches 'zero-knowledge-proof-information' speichert und dem Verifier zusichert, dass der zu verifizierende Nutzer den Claim tatsächlich besitzt.

\section{Sovrin}
\label{sovrin}
Sovrin (Foundation) \footnote{https://sovrin.org/} ist eine gemeinnützige Organisation, die etabliert wurde, um das Governance-Framework 'Sovrin Network' zu administrieren, welches ein öffentlicher Dienstleistungsservice ist zum Ermöglichen der SSI \cite{ID39}. Dabei ist die Funktion der Sovrin Foundation das Sicherstellen des öffentlichen und global zugänglichen Sovrin-Identity-Systems.
Der verteilte Speicher des Netzwerks ist die eigen entwickelte Blockchain, die vollständig darauf abzielt SSI zu implementieren. Folgende vier Eigenschaften seien für ein erfolgreiches SSI-System notwendig und wurden in Sovrin Network eingebaut \cite{ID40}:
\begin{itemize}
	\item Governance: Alle Stakeholder können dem Netzwerk vollständig vertrauen
	\item Performance: Das Netzwerk soll auf dem gleichen Level skalieren wie das Internet selber
	\item Zugänglichkeit: Das Netzwerk ist für jeden zugänglich
	\item Privatsphäre: Die sichersten Standards werden implementiert
\end{itemize}
Ein Mechanismus zum verbessern der Privatsphäre ist das 'pseudonymous by default'-Konzept, welches Sovrin umsetzt. Hierbei wird für jede Relation (beispielsweise PersonX->Arbeitgeber, PersonX->Verkäufer, etc) eine eigene DID verwendet. Somit ist mehr Privatsphäre gegeben und auch die Sicherheit ist verbessert. \\
Die Sovrin-Blockchain ist eine sog 'permissioned' Blockchain, was bedeutet, dass ein Knoten, der Transaktionen tätigen oder validieren möchte (also ein sog. 'Validator-Knoten') eine Erlaubnis braucht. Organisationen, die diese Erlaubnis erhalten haben nennen sich bei Sovrin 'Stewards'. Diese sind global verteilt und aktuell gibt es 50 Stewards in 13 Ländern und 6 Kontinenten \cite{ID43}. 
\subsection{Technische Grundlagen}
Sovrin versucht sich an DNS\footnote{Domain Name System} zu orientieren, da DNS knapp eine Milliarde Einträge hat und über 100 Milliarden anfragen pro Tag bearbeitet \cite{ID42}. Übertragen auf den Identitätskontext - unter der Annahme dass Identitäten mehrere DID's besitzten - muss das Netzwerk möglicherweise täglich Trillionen von Anfragen bearbeiten.
Während DNS keinen Konsensus-Algorithmus verwendet ist dies bei der Sovrin-Blockchain jedoch nicht der Fall. Um dennoch eine gute Skalierung zu implementieren werden zwei verschiedene Arten von Knoten im Netzwerk verwendet:
\begin{itemize}
	\item \textbf{Validator-Knoten}: Ist eine kleine Mengen an Knoten im Netzwerk deren Funktion es ist Transaktionen zu akzeptieren
	\item \textbf{Observer-Knoten}: Eine größere Menge an Knoten, die Lese-Anfragen bearbeiten
\end{itemize}
Issuer, Verifier und Holder erreichen diese Knoten über Agenten. Diese Agenden können beispielsweise Mobilanwendungen sein und die Aufgabe haben mit dem Sovrin-Network in Verbindung zu stehen. Agenten können Identitätstransaktionen stellvertretend für den Identitätsträger tätigen und kommunizieren direkt mit anderen Agenten. Dies funktioniert, da der Agent Zugriff auf den privaten Schlüssel hat. Demnach können beispielsweise DID-Dokumente modifiziert werden oder Transaktionen getätigt werden.
Zudem werden private Daten (wie Claims) ebenso auf dem Agenten gespeichert, während öffentliche Daten auf der Blockchain abgelegt werden. \\
Ähnlich wie bei PolygonId kann von ZK-Proofs Gebrauch gemacht werden oder Anfragen modelliert werden.

\section{ShoCard}
Die letzte hier vorgestellte Lösung basiert auf der Bitcoin Blockchain. Das besondere hierbei ist, dass Bitcoin keine Möglichkeit bietet Smartcontracts oder ähnliches zu implementieren. Dennoch gibt es einen Prozess, der SSI in der Blockchain implementiert. Hierbei verschachtelt Shocard die DID, ein existierenden Credential und zusätzliche Identitätsattribute in einer Bitcoin-Transaktion \cite{ID46}. Shocard verwendet einen zentralen Server der folgende drei Vorgänge implementiert:

\begin{itemize}
	\item \textbf{Bootstrapping}: Ist der Prozess bei dem eine neue Shocard erstellt wird. Hierbei erstellt die Shocard-App ein neues asymmetrisches Schlüsselpaar und scannt die Credentials über die Kamera. Die Daten werden im Anschluss verschlüsselt und auf dem Gerät gespeichert. Ein signierter Hash wird im Anschluss in einer Bitcoin Transaktion gespeichert, wobei die erhaltene Transaktionsnummer zu der ShoCardID des Anwenders wird. Ebenso wird diese Information in der App gespeichert. Ist dieser Prozess abgeschlossen ,so kann der Identitätsträger mit Issuern kommunizieren, um weitere Identitätsattribute anzufragen. Dieser Prozess nennt sich 'certification'.
	\item \textbf{Certification}: Um Identitätsattribute zu erhalten muss der Identitätsträger nachweisen, dass er die Daten kennt, die den Hash generiert hatten (wurde in der App persistiert) und den Schlüssel, der zur Signatur verwendet wurde. Als Ergebnis liegt eine neue Transaktion vor, die die Attribute und die ShoCardID gehasht enthält. Da der Provider die Transaktion getätigt hat, muss er die Transaktionsnummer zusammen mit dem signierten Klartext der neuen Attribute mit dem Nutzer teilen. Diese werden erneut in der Applikation lokal gespeichert. Da der Nutzer die Credentials jedoch nicht verlieren möchte im Falle, dass der Zugriff zur Applikation verloren geht, bietet ShoCard die Möglichkeit Credentials verschlüsselt in einem 'Envelop' zu speichern (ein Speicher den ShoCard verwaltet). Der Envelop kennt den Schlüssel zur Verschlüsselung nicht.
	\item Validation: Dieser Prozess findet statt, wenn ein Identitätsträger den Besitzt von Credentials nachweisen möchte. Hierbei gibt der Nutzer die Referenz des Envelops und den zur Verschlüsselung verwendeten Schlüssel. Somit kann der Verifier die Korrektheit überprüfen. Somit wird klar, dass kein ZK-Proof vorliegt, da dem Verifier sämtliche Informationen des Credentials vorliegen.	
\end{itemize}

\subsection{Probleme von ShoCard}
\label{shocard}
Mit der Verwendung von ShoCard gehen jedoch auch Probleme einher. Beispielsweise wird offensichtlich, dass - wenn der \textbf{ShoCard central server} verwendet wird (der die Envelops speichert) - eine Abhängigkeit zu ShoCard existiert. Sollte das Unternehmen verschwinden und die Server offline nehmen, so verlieren Nutzer ihren Zugang zu den Credentials, wenn sie keine lokale Kopie besitzen. Diese Eigenschaft nimmt ShoCard einen Teil seiner Dezentralität. Zudem existiert das Problem, dass ShoCardIDs nur unidirektional identifizieren (Also Identität -> ShoCardId). Es gibt keine dezentrale Registry, die von einer ShoCardID auf eine DID oder etwas Vergleichbarem auflöst. Zudem gibt es noch mehrere weitere Probleme \cite{ID46}.

\section{Vergleich existierender Lösungen}
\label{cha:vergleich Lösungen}
In der Folgenden Tabelle werden die Lösungen von \textbf{Luniverse}, \textbf{Dock}, \textbf{Polygon} und \textbf{Sovrin} miteinander verglichen. Betrachtet wird dabei die Blockchain und dessen Konsensus-Algorithmus, ob ein Knoten ohne zusätzliche Erlaubnis ein Validator werden kann, ob ZK-Proofs verfügbar sind und wo die Credentials gespeichert werden. Im Anschluss werden die Transaktionskosten betrachtet.

\begin{itemize}
	
	
	
	\item Luniverse
	\begin{itemize}
		\item Blockchain: Luniverse-Blockchain
		\item Konsensus-Algorithmus: LPOA
		\item Permissionless?: No
		\item ZK-Proofs?: Yes
		\item Speicherung: Wallet, Blockchain
	\end{itemize}
	
	\item Dock
	\begin{itemize}
		\item Blockchain: Dock-Sidechain
		\item Konsensus-Algorithmus: GRANDPA
		\item Permissionless?: No
		\item ZK-Proofs?: Yes
		\item Speicherung: Wallet, Blockchain
	\end{itemize}
	
	\item PolygonId
	\begin{itemize}
		\item Blockchain: Polygon PoS
		\item Konsensus-Algorithmus: Proof-of-Stake
		\item Permissionless?: Yes
		\item ZK-Proofs?: Yes
		\item Speicherung: Wallet oder WalletSDK, Blockchain
	\end{itemize}
	
	\item Sovrin
	\begin{itemize}
		\item Blockchain: Sovrin-Network
		\item Konsensus-Algorithmus: Plenum
		\item Permissionless?: No
		\item ZK-Proofs?: Yes
		\item Speicherung: WalletSDK, Blockchain
	\end{itemize}
	
	\item Shocard
	\begin{itemize}
		\item Blockchain: Bitcoin-Blockchain
		\item Konsensus-Algorithmus: Proof-of-work
		\item Permissionless?: Yes
		\item ZK-Proofs?: No
		\item Speicherung: Blockchain, Shocard central server, App
	\end{itemize}
\end{itemize}



\section{Transaktionskosten}
Transaktionen auf der Blockchain sind Prozesse, die in den verteilten Speicher schreiben. Da Transaktionen von Validator-Knoten überprüft werden, muss an das Netzwerk eine Gebühr gezahlt werden. Diese variiert je nach Blockchain, Zustand der Blockchain und Auslastung.
\begin{itemize}
	\item Luniverse: Luniverse gibt an, dass keine Transaktionsgebühren existieren.
	\item Dock: Gibt an, dass keine Gebühren anfallen für die Credentialerstellung. Transaktionen für das Erstellen von Schemas oder Widerrufen von Credentials seien sehr gering \cite{ID44}
	\item Polygon: Bei Polygon lassen sich die Transaktionskosten sehr genau berechnen. Dabei hängt es von zwei Faktoren ab, wie teuer die Transaktion wird \cite{ID45}:
	\begin{itemize}
		\item Menge an Gas: Dies ist eine Metrik für die Leistung, die das Netzwerk für das Ausführen der Transaktion zur Verfügung stellt. Im Falle vom Polygon wird die 'Ethereum Virtual Machine' (EVM) verwendet, welche Platten-Verwendung, CPU-Verwendung, etc. misst und so die Menge an Gas berechnet.
		\item Gaspreis: Dieser Faktor hängt von der Netzwerkauslastung ab. Prinzipiell gibt es drei Modelle zwischen denen man entscheiden muss: Standard, Fast und Rapid. Dabei wird aufsteigend die Transaktionsdauer geringer (30-10 bei Standard und 5-10 bei Rapid). Analog dazu steigt auch der Gaspreis. Auf PolygonScan\footnote{https://polygonscan.com/gastracker} können die Gaspreise historisch betrachtet werden.
	\end{itemize}
	\item Sovrin: Credential und DID-Erstellung sind kostenlos. Für alles andere (also Revokation, Schemas, etc) gibt es jeweils für TestNet und MainNet eigene Preise.
	\item ShoCard: Transaktionskosten auf der Bitcoin Blockchain werden in Satoshi ($1 * 10^{-6} Bitcoin$) pro Byte berechnet. Demnach kostet die Transaktion mehr, je nachdem viele Daten in die Blockchain geschrieben werden. Im Jahr 2023 betrugen die Kosten für eine Transaktion im Durchschnitt zwischen einen und drei Euro \cite{ID49}. Es ist jedoch anzunehmen, dass Transaktionen, die Data-Anchoring betreiben, mehr kosten, da mehr Daten geschrieben werden.
	
\end{itemize}

\section{Konsensus-Algorithmus}
Konsensus-Algorithmen werden verwendet, um einen einheitlichen Zustand des Netzwerk zwischen den Knoten festzulegen. Hierbei gibt es verschiedene Ausführungen, die im Folgenden betrachtet werden:
\begin{itemize}
	\item LPOA: Dieses Akronym steht für "Luniverse's Proof of Authority" \cite{ID50}. Hierbei handelt es sich um einen Proof-of-Authority-Algorithmus, wobei ein Validator nicht anhand seiner zur Verfügung gestellten Rechenkraft oder Kryptowährung bemessen wird, sondern an seiner Identität. Potentielle Validatoren (beispielsweise Unternehmen, Institutionen, Regierungen, etc.) müssen sich bei einer Einrichtung oder Organisation, die das PoA-Netzwerk betreibt, bewerben oder eingeladen werden. Diese entscheiden anhand der Reputation und der Vertrauenswürdigkeit des Bewerbers, ob dieser als Validator fungieren darf. Sollte Letzteres erlaubt werden, so werden Verträge verfasst, um die Verantwortlichkeiten im Netzwerk festzulegen. Im Anschluss kann mit dem Validieren von Blöcken begonnen werden. Sollte Fehlverhalten des Validators festgestellt werden, kann dieser aus dem Netzwerk ausgeschlossen werden \cite{ID51}.
	\item GRANDPA: Dieser Algorithmus lässt sich ebenso als Proof-of-Authority-Algorithmus klassifizieren
	\item Proof-of-Stake: Proof of Stake (PoS) ist ein Konsensusmechanismus in Blockchain-Netzwerken, der verwendet wird, um Transaktionen zu validieren und neue Blöcke zur Blockchain hinzuzufügen. Im Gegensatz zu Proof of Work (PoW), bei dem Miner rechenintensive Aufgaben lösen müssen, um Blöcke zu erstellen, basiert PoS auf dem Konzept des 'Stakings' oder des Einsatzes von Kryptowährungen.
	\begin{enumerate}
		\item \textbf{Validatoren und Staking:} In einem PoS-Netzwerk gibt es keine Miner im herkömmlichen Sinne. Stattdessen gibt es Validatoren. Um ein Validator zu werden, müssen Benutzer eine bestimmte Menge der Kryptowährung des Netzwerks als Einsatz hinterlegen. Dieser dient als Garantie dafür, dass der Validator korrekt arbeitet.
		
		\item \textbf{Blockvalidierung:} Wenn eine Transaktion im Netzwerk eingereicht wird, wird ein Validator zufällig ausgewählt, um die Transaktion zu validieren und einen neuen Block hinzuzufügen. Die Wahrscheinlichkeit, ausgewählt zu werden, hängt oft von der Menge des gestakten Vermögens ab, was bedeutet, dass Benutzer mit größeren Stakes eine höhere Chance haben, ausgewählt zu werden.
		
		\item \textbf{Belohnungen:} Validatoren, die korrekt arbeiten und Transaktionen ordnungsgemäß validieren, erhalten Belohnungen in Form von Transaktionsgebühren und neuen Kryptowährungseinheiten, die dem System hinzugefügt werden. Diese Belohnungen werden oft proportional zur Höhe des gestakten Vermögens des Validators verteilt.
		
		\item \textbf{Bestrafungen:} Wenn ein Validator betrügerisches Verhalten zeigt oder versucht, das Netzwerk zu schädigen, kann er seine gestakten Vermögenswerte verlieren.
	\end{enumerate}
	PoS \cite{ID53}bietet mehrere Vorteile, darunter eine geringere Umweltauswirkung im Vergleich zu PoW, da keine rechenintensiven Aufgaben erforderlich sind und eine höhere Skalierbarkeit. Trotz der vielen Vorteile von Proof-of-Stake gibt es auch einige Nachteile \cite{ID53}:
	
	\begin{enumerate}
		\item \textbf{Zentralisierungstendenzen:} PoS kann zu einer gewissen Zentralisierung führen, da Teilnehmer mit großen Stakes mehr Einfluss haben und wahrscheinlicher ausgewählt werden, um Transaktionen zu validieren und Blöcke hinzuzufügen. Dies könnte zu einer Konzentration der Netzwerkvalidierungsmacht führen.
		
		\item \textbf{Reichtumsungleichheit:} PoS belohnt Benutzer proportional zu ihren gestakten Vermögenswerten. Dies könnte die bestehende Reichtumsungleichheit in Kryptowährungen weiter verstärken, da reichere Benutzer größere Stakes halten können und somit mehr Belohnungen erhalten.
		
		\item \textbf{Gefahr von Auslagerung (Staking as a Service):} Einige Benutzer könnten ihre Staking-Aktivitäten an Dritte auslagern, um die Belohnungen zu maximieren. Dies könnte dazu führen, dass reiche Benutzer Dritte beauftragen, um ihre Stakes zu verwalten, was die Dezentralisierung gefährden könnte.
		
		\item \textbf{Geringe Anreize für Aktivität:} Einige PoS-Netzwerke könnten Schwierigkeiten haben, Benutzer dazu zu ermutigen, aktiv am Netzwerk teilzunehmen, da sie bereits gestakte Vermögenswerte besitzen und möglicherweise keine zusätzlichen Anreize sehen, aktiv Transaktionen zu validieren.
		
		\item \textbf{Schwierigkeiten bei der Wahl der Validatoren:} Die Auswahl von zuverlässigen und ehrlichen Validatoren kann eine Herausforderung darstellen. Es müssen Mechanismen implementiert werden, um sicherzustellen, dass betrügerische oder bösartige Validatoren erkannt und bestraft werden.
		
		\item \textbf{Sicherheitsprobleme bei geringer Beteiligung:} PoS-Netzwerke könnten anfällig für Angriffe sein, wenn die Beteiligung gering ist und nur wenige Validatoren vorhanden sind. In solchen Fällen könnten Angreifer leichter die Kontrolle über das Netzwerk erlangen.
		
		\item \textbf{Verlust von gestakten Vermögenswerten:} Bei fehlerhaftem Verhalten oder betrügerischen Handlungen können Validatorn Strafen auferlegt werden, einschließlich des Verlusts ihrer gestakten Vermögenswerte.
	\end{enumerate}
	
	\item Proof of Work (PoW) ist ein Konsensusmechanismus, der in mehreren Blockchain-Netzwerken verwendet wird. Er dient dazu, Transaktionen zu validieren, neue Blöcke zur Blockchain hinzuzufügen und das Netzwerk vor verschiedenen Arten von Angriffen zu schützen. Im Folgenden sind die Grundlagen von PoW zu betrachten:
	
	\begin{enumerate}[label=\arabic*.]
		\item \textbf{Transaktionen validieren:} Im Netzwerk werden Transaktionen von Benutzern gesammelt und in einen Pool gestellt, der darauf wartet, in einen neuen Block aufgenommen zu werden. Diese Transaktionen müssen validiert werden, um sicherzustellen, dass sie den Regeln des Netzwerks entsprechen.
		
		\item \textbf{Rätsellösung:} PoW erfordert von den sogenannten Minern, mathematische Rätsel zu lösen, die als "Proof of Work" bezeichnet werden. Diese Rätsel sind so konzipiert, dass sie eine erhebliche Rechenleistung erfordern und gleichzeitig leicht zu überprüfen sind, sobald sie gelöst sind. Minen ist also ein wettbewerbsfähiger Prozess, bei dem die Miner darum konkurrieren, das Rätsel zu lösen.
		
		\item \textbf{Wettbewerb und Belohnungen:} Die Miner verwenden ihre Rechenleistung, um das Rätsel zu lösen. Der erste Miner, der das Rätsel erfolgreich löst, kann einen neuen Block erstellen und ihn mit den validierten Transaktionen füllen. Dieser neue Block wird dann der Blockchain hinzugefügt. Als Belohnung für ihre Arbeit erhalten die Miner neue Kryptowährungseinheiten (z. B. Bitcoin) sowie die Transaktionsgebühren, die von den Benutzern für die Validierung ihrer Transaktionen gezahlt werden.
		
		\item \textbf{Sicherheit und Dezentralisierung:} PoW schützt das Netzwerk, indem es für Angreifer sehr teuer macht, die Mehrheit der Rechenleistung im Netzwerk zu kontrollieren. Je mehr Rechenleistung ein Angreifer benötigt, desto schwieriger wird es, das Netzwerk zu übernehmen. Dies trägt zur Sicherheit und Dezentralisierung bei, da viele Miner weltweit am PoW-Prozess teilnehmen.
		
		\item \textbf{Schwierigkeitsanpassung:} Das PoW-System passt die Schwierigkeit des zu lösenden Rätsels automatisch an die Gesamtrechenleistung des Netzwerks an. Dies stellt sicher, dass die Zeit zwischen der Erstellung neuer Blöcke ungefähr gleich bleibt, unabhängig davon, wie viele Miner im Netzwerk aktiv sind.
	\end{enumerate}
	
	Obgleich PoW mit vielen Vorteilen einhergeht stehen auch Kritikpunkte im Raum: Unter anderem ist PoW höchst ineffizient, da wie bereits erwähnt nur der erste Miner, der das Rätsel löst belohnt wird, obwohl viele andere Miner gleichzeitig am selben Problem arbeiteten. Dadurch werden große Mengen an Rechenleistung verschwendet. 
	
	
\end{itemize}