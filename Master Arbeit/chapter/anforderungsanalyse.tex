\chapter{Anforderungsanalyse}
\label{cha:anforderungsanalyse}
Folgende Anforderungen gelten für das Identitätsmanagementsystem und entsprechen den von der OECD festgelegten Eigenschaften \cite{ID25} \cite{ID26}.
\section{Funktionale Anforderungen}

\begin{itemize}
	\item Widerruf: Informationen müssen widerruflich sein
	\item Überprüfbarkeit: Es muss dem Nutzer möglich sein die Daten über sich zu überprüfen. Dazu gehört: Welche Daten stehen zur Verfügung und warum, Wer hat Zugriff auf diese Daten und wann wurden die Daten in das System eingetragen.
	\item Selektive-Veröffentlichung: Dem Nutzer muss es möglich sein nur einzelne Claims zu veröffentlichen
\end{itemize}

\section{Nicht-Funktionale Anforderungen}

\begin{itemize}
	\item Vertraulichkeit: Eigenschaften einer Identität müssen vor unautorisierten Offenlegung geschützt werden
	\item Integrität: Informationen dürften nur autorisiert modifiziert werden
	\item Non-Replay: Operationen dürfen nicht erneut ausführbar sein
	\item Nichtabstreitbarkeit: Das Senden von Daten durch den Nutzer kann im Nachhinein nicht abgesprochen werden
	\item Diebstahlschutz: Die Daten dürften nicht von Unbefugten lesbar sein
	
\end{itemize}

\section{Technische Anforderungen}
Als technische Anforderung wird lediglich festgelegt, dass eine DLT verwendet werden soll, was in dieser Arbeit durch die Blockchain realisiert wird. Davon abgesehen werden Komponenten und Schnittstellen verwendet, die die oben genannten (Nicht-) funktionalen Anforderungen erfüllen.
Auch gilt sich in dem Design und Implementierung an möglichst viele Standards zu halten:
\begin{itemize}
	\item W3C Standard für Verifiable Credentials: https://www.w3.org/TR/vc-data-model/
\end{itemize}

\section{Use-Case}
Der zu implementierende Use-Case ist, dass eine Autorität in der Lage sein muss einen Führerschein an einen Nutzer auszustellen. Der Führerschein darf für niemand anderen als den Nutzer einsehbar sein. Es muss jedoch, die Möglichkeit geben, dass eine andere Autorität den Führer auf Authentizität überprüft oder herausfinden kann, ob der Führerschein für die richtige Fahrzeugart ist oder wann der Fahrer den Führerschein erhalten hat. Von Bedeutung ist, dass nicht der ganze Führerschein gezeigt werden muss, sondern dass selektiv entweder einzelne Attribute einsehbar gemacht werden oder Anfragen bejaht/verneint werden.