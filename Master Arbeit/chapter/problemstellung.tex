\chapter{Problemstellung}
\label{cha:grundlagen}

Probleme dieser Daten-Silos sind unter anderem, dass der Nutzer die Kontrolle über seine Daten verliert und dem Online-Dienst vertrauen muss. Darunter fallen die Fragen, woher der Nutzer weiß, ob seine Daten wirklich gelöscht werden, wenn er es beantragt oder wie sicher die Speicherung der Daten erfolgt. Auch stellen eine große Anzahl an Silos eine große Angriffsfläche für Hacker dar. So seien in den USA 25 Nutzer pro Minute Opfer von Identitätsdiebstahl [1]. Aber auch für die Online-Dienste ergeben sich zahlreiche Probleme: 18\% der potentiellen Käufer hinterlassen einen gefüllten Warenkorb, der aufgrund von Authentifikationsproblemen nicht gekauft wird. 82\% der Unternehmen geben an Probleme mit Fake-Usern zu haben und weitere Probleme kommen aufgrund der rechtlichen Regulationen hinzu (DSGVO, „Adhesion Contracts“, etc). Alles in einem sind Daten-Silos eine höchst ineffiziente und teure Art der Nutzerverwaltung: Im Vereinten Königreich wird geschätzt, dass sich die Kosten für Datensilos auf 3.7B€ belaufen (auf die USA hochgerechnet sind das 25B€). Zusammen mit den zahlreichen anderen Problemen wird klar, dass das Problem der Identität im Internet ernstzunehmend ist.