\chapter{Einführung}
\label{cha:einfuehrung}

\section{Hintergrund und Motivation}
 
Das Internet hat sich als disruptive Technologie erwiesen, die die Art und Weise, wie Menschen kommunizieren, Informationen teilen und Geschäfte abwickeln, revolutioniert hat. Im Zeitalter des Internets spielt die Identität im virtuellen Raum eine wichtige Rolle. Normalerweise erfordert die Nutzung eines Online-Dienstes eine einmalige Registrierung und im Anschluss für jede Verwendung eine Anmeldung unter Angabe der zuvor festgelegten Login-Daten. Neben den Login-Daten werden meist auch personenbezogene Daten abgefragt. Wenn ein Nutzer nun X verschiedene Online-Dienste verwendet, so werden X mal identische Daten zur Person gespeichert (Adresse, Vorname, Nachname, Geschlecht, etc.). Dieses Verhalten verursacht die Entstehung von Datensilos, die mit mehreren Problemen einhergehen. Nach \cite{ID10} summieren sich die Kosten für die Identitätsdatenspeicherung in den UK auf knapp 4 Billionen Euro und in den USA hochgerechnet auf 22 Billionen Euro. 

Ein weiteres Problem ist die Benutzerfreundlichkeit für den Anwender. Dieser ist gezwungen für jeden Online-Dienst sichere Login-Daten zu selektieren. Sind diese immer identisch, so stellt dies ein Sicherheitsrisiko dar, denn wenn einmalig ein Password kompromittiert ist, sind alle anderen Dienste in Gefahr. Besser wäre demnach für jeden Online-Dienst unterschiedliche Login-Daten zu verwenden, was jedoch das Merken schwer macht.

Darüber hinaus stellt die Identitätsproblematik im Internet einen potenziellen Angriffsvektor dar. Cyberkriminelle können Schwachstellen in den Authentifizierungssystemen ausnutzen, um unbefugten Zugriff auf Konten zu erlangen oder Identitätsdiebstahl zu begehen. Dies birgt Risiken für die Privatsphäre und Sicherheit der Nutzer. In den USA werden 25 Personen pro Minute Opfer von Identitätsdiebstahl, wobei sich die durchschnittlichen Kosten für einen Online-Händler pro gestohlenem Datensatz personenbezogener/sensibler Daten auf 164 USD belaufen (2023) \cite{ID64}. Drei Jahre zuvor lag dieser Wert noch bei 146 USD.

Statistiken \cite{ID11} zeigen, dass 82\% der Unternehmen unter gefälschten Nutzerkonten leiden. Diese Fake-User verursachen nicht nur finanzielle Schäden, sondern können auch den Ruf eines Unternehmens schädigen. Darüber hinaus werden etwa 18\% der Einkaufswagen aufgrund von Problemen mit den Anmeldedaten aufgegeben. Dies führt zu Umsatzeinbußen für Unternehmen und frustriert potentielle Kunden.

Ein weiteres Problem ist, dass ein Nutzer im Status Quo keine Macht über seine Daten besitzt. Er ist stets davon abhängig dem Anbieter zu vertrauen, dass die Daten bei Anfrage gelöscht werden, sicher gespeichert sind, nicht ungefragt weitergegeben werden, etc. Ebenso ist keinerlei Transparenz darüber gegeben, wofür die Daten im Detail verwendet oder wofür sie gebraucht werden. Alles in einem besteht keine Autonomie für den Nutzer über die Daten, die er bei einem Online-Service angeben muss.

Angesichts dieser Herausforderungen ist die Notwendigkeit einer verbesserten Identitätsverwaltung im Internet offensichtlich. Es werden Lösungen erforscht, die auf dezentralen Identitätsplattformen und Blockchain-Technologie basieren. Solche Ansätze könnten dazu beitragen, die Sicherheit, Privatsphäre und Benutzerfreundlichkeit im Internet zu verbessern, indem sie eine effizientere und sicherere Möglichkeit bieten, Identitäten zu verwalten und zu überprüfen.

\section{Zielsetzung der Arbeit}
\label{zielsetzung}
Als Ziel soll ein Konzept erarbeitet werden, dass dem Nutzer erlaubt Herrscher seiner
Daten zu sein. Er soll eigenständig in der Lage sein Informationen hinzuzufügen, zu
teilen oder Anfragen zu beantworten. Als technologische Grundlage soll DLT (Distributed Ledger Technology) verwendet werden. Es soll ein Prototyp entwickelt werden, der obere Logik implementiert. Es sollen Funktionen zur Verfügung stehen zum:
\begin{itemize}
	\item Identitäten erstellen
	\item Dokumente erstellen
	\item Dokumente überprüfen
	\item Dokumente widerrufen
\end{itemize}
Ein möglicher Anwendungsfall wäre, das ein Nutzer ein Dokument von einer Behörde ausgestellt bekommt und dieses dadurch besitzt. Das Wichtige hierbei ist, dass nur der Nutzer Zugriff zum Dokument und den darin enthaltenen Informationen hat. Zudem muss die Möglichkeit bestehen, dass eine andere Instanz die Korrektheit oder Attribute des Dokumentes überprüfen kann.

Das zu implementierende Szenario ist, dass eine Person seinen Führerschein erhalten möchte. Hierfür meldet er sich bei einer Organisation, die ihm daraufhin einen QR-Code zusendet. Dieser Code kann mit einer App gescannt werden, woraufhin der Nutzer den Führerschein als digitales Dokument erhält. Nach einiger Zeit wird der Fahrer von einer anderen Organisation gestoppt, die den Führerschein überprüfen möchte. Dies geschieht, indem sie einen QR-Code vorzeigen, den der Fahrer erneut scannen muss. Im Anschluss erhält der Kontrolleur einen Beweis für den Besitzt des Führerscheins und kann die Attribute ebenso vergleichen (wann wurde der Führerschein ausgestellt, welcher Fahrzeugtyp, etc.). Im Anschluss ist die Kontrolle abgeschlossen.

\section{Forschungsfragen}
\label{forschungsfragen}
Folgende Forschungsfragen werden in dieser Arbeit beantwortet:
\begin{enumerate}
\item Wie erfolgt die Speicherung verschiedener Typen an Daten innerhalb eines Identitätsmanagementsystems?
\begin{enumerate}
	\item Wie kann die technische Machbarkeit erreicht werden, Daten sowohl öffentlich zugänglich als auch privat zu speichern?
	\item Welche kryptografischen Technologien, wie zum Beispiel Hashing oder Verschlüsselung, sind optimal für die sichere Speicherung von Identitätsdaten geeignet?
	\item Welche Sicherheitsmaßnahmen sind im Falle einer Datenkompromittierung zu ergreifen, und welche Wiederherstellungsoptionen sind verfügbar?
	\item Wie lässt sich die sichere und effektive Ungültigmachung von Informationen (Revokation) gewährleisten?
\end{enumerate}

\item Welchen Nutzer oder Service-Mehrwert generiert ein dezentrales Identitätsmanagementsystem basierend auf DLT?
\begin{enumerate}
	\item Welche konkreten Vorteile ergeben sich für Nutzer und Online-Dienste durch die Implementierung des vorgeschlagenen Identitätsmanagementsystems?
	\item Inwiefern trägt das System zur Lösung der Problematik von Fake-Usern bei?
\end{enumerate}

\item Wie erfolgt die Identitätszuordnung innerhalb des Identitätsmanagementsystems?
\begin{enumerate}
	\item Wie kann eine verlässliche Zuordnung von Identitäten zu Personen auf hohem Sicherheitsniveau erreicht werden?
	\item Welche Ansätze können entwickelt werden, um das Problem zu lösen, dass Nutzer möglicherweise verschiedene Identitäten auf verschiedenen Plattformen verwenden möchten?
\end{enumerate}

\item Welche technologischen Spezifika sind im Bezug auf die Blockchain als DLT im Identitätsmanagementsystem zu entscheiden?
\begin{enumerate}
	\item Welcher Konsensus-Algorithmus ist am besten geeignet, um die Anforderungen des entwickelten Identitätsmanagementsystems zu erfüllen?
	\item Wie können private Schlüssel sicher gespeichert werden, um unbefugten Zugriff zu verhindern?
	\item Inwiefern können Identitätsdokumente erfolgreich als NFTs gespeichert werden, und welche Implikationen ergeben sich daraus?
	\item Welche Rolle spielen Zero-Knowledge-Proofs bei der Entwicklung eines sicheren Identitätsmanagementsystems?
\end{enumerate}
\end{enumerate}

\section{Aufbau der Arbeit}
Zunächst werden die Grundlagen dieser Arbeit gesetzt, indem auf die Historie von Identitätsmanagementsystemen eingegangen wird. Dabei werden der zentrale, föderierte, Nutzer-zentrierte und selbstbestimmte Ansatz erläutert. Daraufhin werden die Grundlagen von Distributed Ledger Technology erläutert und deren Bedeutung für Identitätsmanagementsysteme angeführt. Im Anschluss werden die Anforderungen formuliert - die in funktionale/nicht-funktionale und technische Anforderungen gegliedert sind - und existierende Lösungen zunächst vorgestellt und im nächsten Schritt verglichen. Im Fokus stehen hierbei Luniverse, Dock, Sovrin, ShoCard und PolygonId. Nachdem erklärt wurde, warum Polygon die passende Plattform ist, wird ein System-Design vorgestellt, welches im nächsten Schritt implementiert wird. Hierbei wird zunächst das Grobe Design skizziert und im Anschluss jede Komponente einzeln illustriert. Im vorletzten Schritt werden Metriken festgelegt (Laufzeit,Sicherheit) und evaluiert. Zum Abschluss wird ein Fazit verfasst, es wird überprüft, ob alle Anforderungen erfüllt wurden und eine Diskussion findet statt. Auch werden zukünftige Forschungsrichtungen aufgeführt und der Beitrag zur Forschung illustriert.