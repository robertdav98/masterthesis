\chapter{Einführung}
\label{cha:einfuehrung}

\section{Hintergrund und Motivation}
 
Das Internet hat sich als disruptive Technologie erwiesen, die die Art und Weise, wie Menschen kommunizieren, Informationen teilen und Geschäfte abwickeln, revolutioniert hat. Im Zeitalter des Internets spielt die Identität im virtuellen Raum eine wichtige Rolle. Normalerweise erfordert die Nutzung eines Online-Dienstes eine einmalige Registrierung und im Anschluss für jede Verwendung eine Anmeldung unter Angabe der zuvor festgelegten Login-Daten. Neben den Login-Daten werden meist auch personenbezogene Daten abgefragt. Wenn ein Nutzer nun X verschiedene Online-Dienste verwendet, so werden X mal identische Daten zur Person gespeichert (Adresse, Vorname, Nachname, Geschlecht, etc.). Dieses Verhalten verursacht die Entstehung von Datensilos, die mit mehreren Problemen einhergehen. Nach \cite{ID10} summieren sich die Kosten für die Identitätsdatenspeicherung in den UK auf knapp 4 Billionen Euro und in den USA hochgerechnet auf 22 Billionen Euro. 

Ein weiteres Problem ist die Benutzerfreundlichkeit für den Anwender. Dieser ist gezwungen für jeden Online-Dienst sichere Login-Daten zu selektieren. Sind diese immer identisch, so stellt dies ein Sicherheitsrisiko dar, denn wenn einmalig ein Password kompromittiert ist, sind alle anderen Dienste in Gefahr. Besser wäre demnach für jeden Online-Dienst unterschiedliche Login-Daten zu verwenden, was jedoch das Merken schwer macht.

Darüber hinaus stellt die Identitätsproblematik im Internet einen potenziellen Angriffsvektor dar. Cyberkriminelle können Schwachstellen in den Authentifizierungssystemen ausnutzen, um unbefugten Zugriff auf Konten zu erlangen oder Identitätsdiebstahl zu begehen. Dies birgt Risiken für die Privatsphäre und Sicherheit der Nutzer. In den USA werden 25 Personen pro Minute Opfer von Identitätsdiebstahl, wobei sich die durchschnittlichen Kosten für einen Online-Händler pro gestohlenem Datensatz personenbezogener/sensibler Daten auf 165 USD belaufen \cite{ID10}. Im Vorjahr 2015 betrug die Menge 105 USD.

Statistiken \cite{ID11} zeigen, dass 82\% der Unternehmen unter gefälschten Nutzerkonten leiden. Diese Fake-User verursachen nicht nur finanzielle Schäden, sondern können auch den Ruf eines Unternehmens schädigen. Darüber hinaus werden etwa 18\% der Einkaufswagen aufgrund von Problemen mit den Anmeldedaten aufgegeben. Dies führt zu Umsatzeinbußen für Unternehmen und frustriert potentielle Kunden.

Ein weiteres Problem ist, dass ein Nutzer im Status Quo keine Macht über seine Daten besitzt. Er ist stets davon abhängig dem Anbieter zu vertrauen, dass die Daten bei Anfrage gelöscht werden, sicher gespeichert sind, nicht ungefragt weitergegeben werden, etc. Ebenso ist keinerlei Transparenz darüber gegeben, wofür die Daten im Detail verwendet oder wofür sie gebraucht werden. Alles in einem besteht keine Autonomie für den Nutzer über die Daten, die er bei einem Online-Service angeben muss.

Angesichts dieser Herausforderungen ist die Notwendigkeit einer verbesserten Identitätsverwaltung im Internet offensichtlich. Es werden Lösungen erforscht, die auf dezentralen Identitätsplattformen und Blockchain-Technologie basieren. Solche Ansätze könnten dazu beitragen, die Sicherheit, Privatsphäre und Benutzerfreundlichkeit im Internet zu verbessern, indem sie eine effizientere und sicherere Möglichkeit bieten, Identitäten zu verwalten und zu überprüfen.

\section{Zielsetzung der Arbeit}
\label{zielsetzung}
Als Ziel soll ein Konzept erarbeitet werden, dass dem Nutzer erlaubt Herrscher seiner
Daten zu sein. Er soll eigenständig in der Lage seine Informationen hinzuzufügen, zu
teilen oder Anfragen zu beantworten. Als technologische Grundlage soll DLT (Distributed Ledger Technology) verwendet werden. Es soll ein Prototyp entwickelt werden, der obere Logik implementiert. Es sollen Funktionen zur Verfügung stehen zum:
\begin{itemize}
	\item Identitäten erstellen
	\item Dokumente erstellen
	\item Dokumente überprüfen
	\item Dokumente widerrufen
\end{itemize}
Ein möglicher Anwendungsfall wäre, das ein Nutzer ein Dokument von einer Behörde ausgestellt bekommt und dieses dadurch besitzt. Das Wichtige hierbei ist, dass nur der Nutzer Zugriff zum Dokument und den darin enthaltenen Informationen hat. Zudem muss die Möglichkeit bestehen, dass eine andere Instanz die Korrektheit oder Attribute des Dokumentes überprüfen kann.

\section{Forschungsfragen}
\label{forschungsfragen}
Folgende Forschungsfragen werden in dieser Arbeit beantwortet:
\begin{enumerate}
	\item Ist es möglich Daten privat aber öffentlich zu speichern?
	\item Wie werden die Daten gespeichert? (Hash, Verschlüsselt, etc.) 
	\item Welcher Mehrwert wird generiert für den User und die Online-Dienste?
	\item Kann das Problem der Fake-user hiermit gelöst werden?
	\item Was passiert im Falle einer Kompromittierung?  Recovery-Optionen?
	\item Wie sollen Informationen wieder ungültig gemacht werden? (Revokation)
	\item Wie kann sichergestellt werden, dass die Identität wirklich der Person zuzuordnen ist?
	\item Wie soll das Problem gelöst werden, dass ein Nutzer evtl. verschiedene Identitäten auf verschiedenen Plattformen verwenden möchte (Reddit-Account-Identität vs Online-Banking-Account)
	\item Blockchain-Forschungsfragen:
	\begin{enumerate}
		\item Welcher Konsensus-Algorithmus ist am passendsten für das entwickelte Identitätsmanagementsystem?
		\item Soll eine private oder eine öffentliche Blockchain verwendet werden?
		\item Wie und wo werden die privaten Schlüssel speichern?
		\item Sollen 'permissioned' oder 'permissionless' Blockchains verwendet werden?
		\item Können die Dokumente als NFT's gespeichert werden?
		\item Welche Rolle spielen Zero-Knowledge-Proofs für die Entwicklung eines Identitätsmanagementsystems?
	\end{enumerate}	
\end{enumerate}

\section{Aufbau der Arbeit}
Zunächst werden die Grundlagen dieser Arbeit gesetzt, indem auf die Historie von Identitätsmanagementsystemen eingegangen wird. Daraufhin werden die Grundlagen von Distributed Ledger Technology erläutert und deren Bedeutung für Identitätsmanagementsysteme angeführt. Im Anschluss werden die Anforderungen formuliert und existierende Lösungen zunächst vorgestellt und im nächsten Schritt verglichen. Nachdem erklärt wurde, warum Polygon die passende Plattform ist, wird ein System-Design vorgestellt, welches im nächsten Schritt implementiert wird. Im vorletzten Schritt werden Metriken festgelegt und evaluiert. Zum Abschluss wird ein Fazit verfasst.