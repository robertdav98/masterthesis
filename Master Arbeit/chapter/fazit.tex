\chapter{Fazit}
\label{cha:fazit}

\section{Zusammenfassung der Arbeit}
Diese Thesis nimmt eine tiefgehende Betrachtung der Problematik vor, die mit herkömmlichen Identitätsmanagementsystemen in der heutigen vernetzten Welt einhergeht. Sie identifiziert und analysiert den wachsenden Bedarf für SSI und legt einen besonderen Fokus auf die technischen Grundlagen dieses Ansatzes.

Zu Beginn wurden die Grundlagen und die historische Entwicklung von Identitätsmanagementsystemen beleuchtet. Diese umfassen sowohl zentrale als auch dezentrale Ansätze, wobei letztere im Kontext von SSI von besonderem Interesse sind. Die Thesis beschäftigt sich ausführlich mit den technischen Grundlagen von SSI, einschließlich der Konzepte von Holder (Identitätsinhaber), Issuer (Identitätsaussteller) und Verifier (Identitätsüberprüfer). Dieser theoretische Rahmen bildet die Grundlage für das Verständnis und die Implementierung von SSI.

Ein zentraler Aspekt dieser Arbeit ist die Bedeutung von Distributed Ledger Technology (DLT) für die Umsetzung von SSI. Die Verwendung von DLT ermöglicht es, Identitätsdaten sicher und dezentral zu speichern, wodurch die Kontrolle über persönliche Informationen wieder in die Hände der Nutzer gelegt wird. Die Thesis untersucht verschiedene DLT-Plattformen und deren Eignung für die Implementierung von SSI, wobei die Wahl auf Polygon als Technologie fällt.

Das Design des SSI-Prototyps wird im Detail erläutert, wobei die Interaktion zwischen den verschiedenen Komponenten des Systems eine entscheidende Rolle spielt. Ein spezifisches Szenario wird ausgewählt und umgesetzt, um die praktische Anwendung von SSI aufzuzeigen.

Eine umfassende Evaluierung des Prototyps erfolgt anhand verschiedener Metriken, darunter die Laufzeit und Sicherheit. Neben den Metriken wird auch die Erfüllung der Anforderungen im Detail betrachtet.

\section{Erfüllung der Zielsetzung}
Die in Kapitel \ref{zielsetzung} gesetzte Zielsetzung  wurde vollständig erfüllt. Es wurde - wie beschrieben - ein dezentrales Identitätsmanagementsystem entwickelt, welches DLT implementiert. Hierbei ist - wie gefordert - der Nutzer Herrscher über seine Daten. Sowohl die funktionalen/nicht-funktionalen als auch technischen Anforderungen wurden erfüllt. Zudem sind Sicherheit, Kontrollierbarkeit und Skalierbarkeit berücksichtigt worden. Auch das vorgestellte Szenario kann ausgeführt werden - implementiert wurde es doch Anhand eines Führerschein-Beispiels.


\section{Diskussion}
In dieser Arbeit gab es mehrere Optionen das in der Zielsetzung beschriebene Ziel zu implementieren. Wäre die Anwendung beispielsweise nicht im Polygon Framework implementiert worden, so wäre Dock eine valide Alternative gewesen. Der Vorteil hierbei ist, dass keine eigenen Issuer-Knoten oder ähnliches gehostet werden müssen und viele Dienste bereits implementiert wurden, wie das Zuschicken von Credentials über Email - direkt an mehrere Empfänger. Auch ist ein Vorteil, dass die Dock-REST-API viele SSI-Methoden bereits anbietet, wie das Erstellen und Widerrufen von Credentials, Erstellen von Schemas, etc.\\
Auch wäre es in der implementierten Lösung bedenkenswert den Issuer oder Verifier nicht off-chain sondern on-chain zu implementieren. Hierbei befände sich die Logik im Smartcontract und gibt die Möglichkeit Operationen direkt in der Blockchain durchzuführen (Token-Transfer, Verwendung anderer Smart-Contracts, bessere Nachverfolgbarkeit, etc.). Auch wäre es denkbar einen eigenen Identity-Wallet zu implementieren und zu verwenden. In der hier vorgestellten Lösung wurde die PolygonId-App verwendet, jedoch stehen mehrere SDK zur Verfügung, womit ein Entwickler seinen eigenen Wallet erstellen kann.\\
Zusätzlich gibt es das Konzept der Non-Fungible-Token, die mit dem Proof-Of-Ownership-Gedanken dem SSI-Konzept ähneln. Daher wäre es abwägbar, ob eine Integration von NFT's in eine weiterführende Arbeit von Bedeutung wäre.


\section{Zukünftige Forschungsrichtungen}
Der Blick in die Zukunft des Forschungsfelds im Bereich SSI und deren Implementierung mithilfe von Distributed Ledger Technology (DLT) verspricht faszinierende Entwicklungen. In den kommenden Jahren werden Forscher voraussichtlich verstärkt die Skalierbarkeit von SSI-Systemen erforschen, um diese für breitere Anwendungsbereiche tauglich zu machen. Ein Schwerpunkt könnte dabei auf der Integration von SSI in bestehende digitale Infrastrukturen und Plattformen liegen, um die nahtlose Interoperabilität zu gewährleisten.

Des Weiteren wird die Verbesserung der Sicherheitsaspekte von SSI von entscheidender Bedeutung sein. Forschung wird sich auf fortschrittliche Kryptographie, Authentifizierungsmethoden und Datenschutzkonzepte konzentrieren, um das Vertrauen in diese Systeme zu stärken und Datenschutzverletzungen zu minimieren.

Die Entwicklung von internationalen Standards und Protokollen für SSI wird eine weitere wichtige Forschungsrichtung sein. Diese Standards sind entscheidend, um die weltweite Akzeptanz und Anwendung von SSI-Systemen zu fördern und Interoperabilität zwischen verschiedenen Implementierungen sicherzustellen.

Schließlich werden auch soziale und ethische Aspekte der Selbstsouveränen Identitäten verstärkt in den Fokus rücken. Forschung wird sich auf Fragen der Akzeptanz, der Bildung und Sensibilisierung der Nutzer, sowie auf die ethischen Implikationen in Bezug auf Identitätsmanagement und Datenschutz konzentrieren.

Insgesamt versprechen zukünftige Forschungsrichtungen auf dem Gebiet der Selbstsouveränen Identitäten und DLT eine spannende und vielversprechende Entwicklung, die sowohl technische als auch gesellschaftliche Herausforderungen angehen wird, um die Vision von selbstsouveränen und sicheren Identitäten in einer digitalisierten Welt voranzubringen.

\section{Beitrag zur Forschung}
Diese Arbeit leistet deutende Beiträge zur Forschung im Bereich 'Dezentrale Identitätsmanagementsysteme', da zunächst mehrere existierende Lösungen vorgestellt und verglichen wurden. Dabei wurden unterschiedliche Konzepte zur Realisierung von SSI vorgestellt und im Anschluss verglichen.  Diese Art von wissenschaftlicher Beitrag existiert so noch nicht. Auch wurde eine Realisierung von einem der vorgestellten Konzepte implementiert mit ausführlichen Erläuterungen. Ebenso wurde Wissen zusammengefasst, dass relevante Blockchain-Konzepte beschreibt, ohne irrelevante Informationen oder mathematische Details zu thematisieren.