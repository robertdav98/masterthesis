%----------------- KONFIGURATION ----------------- %
\pagestyle{empty} % enthalten keinerlei Kopf oder Fuß


\chapter*{\centerline{Abstract}} % (fold)
\label{cha:abtract}

\section*{\centerline{Deutsch}}
Die vorliegende Masterarbeit beschäftigt sich mit dem Konzept der Self-Sovereign-Identity (SSI) und hat das Ziel einen Prototypen für ein Identitätsmanagementsystem zu implementieren, welches auf Distributed-Ledger-Technology (DLT) basiert. Diese Arbeit führt mehrere SSI-Lösung (Sovrin, Dock, ShoCard, Luniverse, PolygonId) auf und vergleicht diese im Anschluss miteinander. Als Ergebnis des Vergleichs tritt PolygonId als beste Option hervor. Demnach wird der Prototyp basierend auf PolygonId implementiert. Die anschließende Analyse ergibt, dass die Anwendung voll funktional ist und Operationen (DID-Erstellung, Claim-Erstellung, Erhalten des Claims als QR-Code, Identitäten lesen und Claims widerrufen) eine Laufzeit von 1 bis 1.5 Sekunden haben. Zudem ergab die anschließende STRIDE-Analyse, dass die Anwendung resistent gegen eine Vielzahl an Angriffskategorien ist. 

\section*{\centerline{English}}

The present master's thesis focuses on the concept of Self-Sovereign Identity (SSI) and aims to implement a prototype for an identity management system based on Distributed Ledger Technology (DLT). This work introduces several SSI solutions (Sovrin, Dock, ShoCard, Luniverse, PolygonId) and subsequently compares them. The comparison results highlight PolygonId as the optimal choice. Accordingly, the prototype is implemented based on PolygonId. The subsequent analysis reveals that the application is fully functional, with operations (DID creation, claim creation, receiving the claim as a QR code, reading identities, and revoking claims) having a runtime of 1 to 1.5 seconds. Additionally, the subsequent STRIDE analysis indicates that the application is resilient against a variety of attack categories.
