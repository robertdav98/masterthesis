\chapter{Existierende Forschungsrichtungen}
\label{cha:grundlagen}

Im Bereich von SSI (Self-sovereign Identity) gibt es mehrere Forschungsrichtungen. Zum einen wird auf technologischer Sicht aktiv geforscht. Darunter fällt das erforschen und verbessern von kryptographischen Verfahren. Dabei ist das Ziel diese zukunftssicherer, effizienter, innovativer oder verteilt zu gestalten. Auch wird im Bereich der „Organisation“ von SSI geforscht. Darunter fallen Fragestellungen wie man SSI transparenter, zugänglicher, standardisierter und mehrheitstauglicher gestaltet. Vor allem letzter Aspekt stellt ein eigenes Forschungsgebiet dar (User Experience und Usability). Wie bereits in der Einleitung erläutert stellen auch (datenschutz-) rechtliche Aspekte ein erforschungsbedürftiges Konzept dar. Das Letze hier zu erwähnende Forschungsfeld ist die soziale Komponente von SSI und stellt sich die Frage, inwiefern SSI die Gesellschaft ändert (Vertrauen in Technologien, Akzeptanz in der Gesellschaft, Kriminalität, etc.).