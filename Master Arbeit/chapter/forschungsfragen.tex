\chapter{Forschungsfragen}
\label{cha:grundlagen}

\begin{itemize}
	\item Ist es möglich Daten privat aber öffentlich zu speichern?
	\item Wie werden die Daten gespeichert? (Hash, Verschlüsselt, etc.) 
	\item Welcher Mehrwert wird generiert für den User und die Online-Dienste?
	\item Kann das Problem der Fake-user hiermit gelöst werden?
	\item Welche Service-Level-Agreements könnten angegeben werden?
	\item Was passiert im Falle einer Kompromittierung?  Recovery-Optionen?
	\item Wie sollen Informationen wieder ungültig gemacht werden? (Revokation)
	\begin{itemize}
		\item Wieviel kostet der Betrieb?
		\item Wie teuer sind solche Dienste in der Regel für den User/Online-Dienst?
		
	\end{itemize}
	\item Wie kann sichergestellt werden, dass die Identität wirklich der Person zuzuordnen ist?
	\item Wie soll das Problem gelöst werden, dass man evtl. verschiedene Identitäten auf verschiedenen Plattformen verwenden möchte (Reddit-Account-Identität vs Online-Banking-Account)
	\item Blockchain-Forschungsfragen:
		\begin{itemize}
		\item Welcher Consensus-Algorithmus ist am passensten für das entwickelte Identitätsmanagementsystem?
		\item Soll eine private oder eine öffentliche Blockchain verwendet werden?
		\item Wie und wo werden die privaten Schlüssel speichern?
		\item Sollen "permissioned" oder "permissionless" Blockchains verwendet werden?
		\item Sind Authentifikations-Graphen valide Lösungsansätze?
		\item Können die Dokumente als NFT's gespeichert werden?
		\item Welche Rolle spielen Zero-Knowledge-Proofs für die Entwicklung eines Identitätsmanagementsystems?
		\end{itemize}
	
	
	
\end{itemize}