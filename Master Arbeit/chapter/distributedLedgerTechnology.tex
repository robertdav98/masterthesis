\chapter{Distributed Ledger Technology}
\label{cha:distributedLedgerTechnology}

\section{Merkmale und Vorteile von DLT}
Distributed Ledger ist - wie der Titel bereits beschreibt - eine , für diese Arbeit, bedeutende Technologie. Dabei sind folgende Merkmale und Vorteile der DTL relevant \cite{ID19}:
\begin{itemize}
	\item DLT ermöglicht das Betreiben einer hochverfügbaren Datenbank (eines 'Ledgers'), da nicht eine zentrale Instanz für die Verfügbarkeit verantwortlich ist, sondern die Gesamtheit der Knoten in dem Netzwerk.
	
	\item Ebenso ermöglicht die Dezentralität des DLT eine verteilte Speicherung und Verarbeitung.
	
	\item Manipulationsresistenz wird durch kryptographische Verfahren innerhalb der Blockchain sichergestellt. Im Fall  der Blockchain sind sind sind diese in der Regel asymmetrische Verfahren, was bedeutet, dass private/öffentliche Schlüssel zum Ent- oder Verschlüsseln der Daten verwendet werden, wobei 'Integer Factorization', "Discrete Logarithm' oder 'Elliptic Curves' verwendet werden können \cite{ID23}
	
	\item Zensurresistenz kann gewährleistet, indem beispielsweise alle Knoten die gleichen Berechtigungen haben und somit keine machthabende Instanz existiert. Alle Teilnehmer im Netzwerk werden als Knoten (Nodes) bezeichnet und besitzen jeweils eine lokale Kopie des Ledgers. Änderungen werden nun auf der Kopie ausgeführt und im Anschluss in dem Netzwerk synchronisiert. Das Netzwerk gilt als "untrustworthy" (nicht vertrauenswürdig), wenn willkürliche einzelne Knoten sog. 'Byzantine-Failures' \cite{ID20} \cite{ID21} erzeugen können. Dies bedeutet, dass versucht wird beliebig falsches Verhalten im System zu erzeugen (unauthentische Daten, Zusammensturz des Systems, etc). Der Resistenzgrad des Netzwerks gegenüber diesen Angriffen wird als 'Byzantine-Toleranz' bezeichnet und wird in der Regel durch Abstimmungen im Netzwerk (Beispielsweise Konsensus-Algorithmen) verhindert. Beispiele im Blockchain-Kontext sind 'Proof-of-Work' oder "Proof-of-Stake' \cite{ID22} Algorithmen.
	
	\item Möglichkeit zur 'Demokratisierung' von Daten: Durch DLT kann ermöglicht werden, dass Individuen und/oder Organisationen kooperativ Kontrolle über Daten ausüben
	
\end{itemize}

\section{Anwendung von DLT im Bereich der digitalen Identität}
Die oben genannten Eigenschaften sind für das Betreiben eines Identitätsmanagementsystems optimal, da diese hochverfügbar sein sollten, mit möglichst kurzen oder nicht existierenden Downtimes. Auch ist eine verteilte Speicherung und Verarbeitung eine effiziente Möglichkeit große Menge an Anfragen zu bearbeiten oder eine Vielzahl an Identitätsdaten zu speichern. Zusätzlich ist Manipulationsresistenz von großer Bedeutung, da die Identitätsdaten stets authentisch sein müssen, um beispielsweise Dokumentenfälschung oder Identitätsdiebstahl zu vermeiden. Ebenso können finanzielle Transaktionen hiermit abgewickelt werden, was in der analogen Welt oft im Zusammenhang mit der Dokumentenausstellung stattfinden. Ein Beispiel hierfür sind die Gebühren beim Beantragen eines Reisepasses oder die Strafgebühr für das zu späte Neubeantragen eines Abgelaufenen Ausweises.
Die Möglichkeit sog. 'smart contracts' - also eigene Programme - zu schreiben ist eine Eigenschaft, die nicht in allen DLT's gegeben ist. Dennoch wird diese Eigenschaft an dieser Stelle erwähnt, da einige Blockchains wie Ethereum Letzteres unterstützen und somit einem Software-Entwickler die Chance geben fehlende Software im Identitätsmanagementsystems zu implementieren.

Diese Merkmale von DLT machen es zu einer idealen Technologie für die Umsetzung von Self-Sovereign-Identity. Sie ermöglicht eine sichere, vertrauenswürdige und selbstbestimmte Verwaltung von Identitätsinformationen, wodurch Benutzer die Kontrolle über ihre Identität zurückerlangen und die Notwendigkeit von zentralen, vertrauenswürdigen Dritten verringert wird.
