\chapter{System-Design}
\label{cha:systemdesign}

\section{Architektur des dezentralen Identitätsmanagementsystems}

\subsection{Entscheidung über Framework}
Um sich für eine Architektur festzulegen, muss zunächst die Entscheidung über die darunterliegende Plattform getroffen werden. Dabei stehen die in Kapitel 5 betrachteten Lösungen zur Auswahl: Luniverse, Dock, PolygonId, Sovrin und Shocard. Für die Entwicklung des Prototypen wird im folgenden :
\begin{itemize}
	
	\item Polygon (als unterliegende Blockchain) zeigt folgende Vorteile auf \cite{ID54}:
	\begin{itemize}
		\item Die Blockchain skaliert hervorragend mit einer steigenden Anzahl von Transaktionen
		\item Geringe Transaktionskosten
		\item Hohe Interoperabilität mit Ethereum
		\item Etablierte Plattform und weite Verbreitung im Markt
	\end{itemize}
	
	\item PolygonId:
	\begin{itemize}
		\item Möglichkeit zum Widerruf von Informationen gegeben
		\item Informationen sind überprüfbar
		\item Selektive-Disclosure implementierbar
		\item Alle nicht-funktionalen Anforderungen implementierbar
		\item Unterstützt W3C Standard für VC's
		\item Credential Exchange erfolgt nach Identity-Foundation Standard
	\end{itemize}
\end{itemize}
Prinzipiell besteht die Architektur aus drei Komponenten: Einem Verifier, einem Issuer und dem Holder. Jede dieser drei Komponenten laufen unabhängig voneinander und können mit fremd-implementierten Instanzen kommunizieren. 
\section{Komponenten und deren Funktionalitäten}

\section{Interaktion zwischen den Komponenten}
\blindtext

\section{Integration von DLT in das Systemdesign}
\blindtext
