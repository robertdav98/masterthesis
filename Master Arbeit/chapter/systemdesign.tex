\chapter{System-Design}
\label{cha:systemdesign}

\section{Architektur des dezentralen Identitätsmanagementsystems}

\subsection{Entscheidung über Framework}
Um sich für eine Architektur festzulegen muss zunächst die Entscheidung über die darunterliegende Plattform getroffen werden. Dabei stehen die in Kapitel 5 betrachteten Lösungen zur Auswahl: Luniverse, Dock, PolygonId, Sovrin und Shocard. Die für den Prototypen ausgewählte Plattform wird PolygonId aus folgenden Gründen:
\begin{itemize}
	
	\item Polygon (als unterliegende Blockchain) zeigt folgende Vorteile auf \cite{ID54}:
	\begin{itemize}
		\item Die Blockchain skaliert hervorragend mit einer steigenden Anzahl von Transaktionen
		\item Geringe Transaktionskosten
		\item Hohe Interoperabilität mit Ethereum
		\item Etablierte Plattform und weite Verbreitung im Markt
	\end{itemize}
	
	\item PolygonId:
	\begin{itemize}
		\item Möglichkeit zum Widerruf von Informationen gegeben
		\item Informationen sind überprüfbar
		\item Selektive-Disclosure implementierbar
		\item Alle nicht-funktionalen Anforderungen implementierbar
		\item Unterstützt W3C Standard für VC's
		\item Credential Exchange erfolgt nach Identity-Foundation Standard
	\end{itemize}
\end{itemize}

\section{Komponenten und deren Funktionalitäten}
Das Identitätsmanagementsystem kann ohne bestimmte Komponenten nicht funktionieren. Es ist definitiv ein DID-Resolver notwendig: Also eine Komponente, die DID's auf deren DID-Dokumente ableitet.

\section{Interaktion zwischen den Komponenten}
\blindtext

\section{Integration von DLT in das Systemdesign}
\blindtext
