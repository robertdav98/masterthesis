\chapter{Codebeispiele}
\label{cha:codebeispiele}

\begin{lstlisting}[language = Java , frame = trBL , firstnumber = last , escapeinside={(*@}{@*)}]
	public class Factorial
	{
		public static void main(String[] args)
		{   final int NUM_FACTS = 100;
			for(int i = 0; i < NUM_FACTS; i++)
			System.out.println( i + "! is " + factorial(i));
		}
		
		public static int factorial(int n)
		{   int result = 1;
			for(int i = 2; i <= n; i++) (*@\label{for}@*)
			result *= i;
			return result;
		}
	}
\end{lstlisting}

And you can reference line \ref{for} in the code!